\documentclass[11pt]{article}

\usepackage{babel}
\usepackage{xcolor}
\usepackage{graphicx}
\usepackage[top=0.6in,bottom=0.6in,right=1in,left=1in]{geometry}

\usepackage[utf8]{inputenc}
\usepackage[T1]{fontenc}
\usepackage{longtable}
\usepackage{wrapfig}
\usepackage{listings}
\usepackage{rotating}
\author{Guelfen Abdelheq}
\date{\today}
\title{Documentation}

\definecolor{codegreen}{rgb}{0,0.6,0}
\definecolor{codegray}{rgb}{0.5,0.5,0.5}
\definecolor{codepurple}{rgb}{0.58,0,0.82}
\definecolor{backcolour}{rgb}{0.95,0.95,0.92}
\definecolor{darkblue}{RGB}{31,56,100}

\lstdefinestyle{PythonStyle}{
    backgroundcolor=\color{backcolour},
    commentstyle=\color{codegreen},
    keywordstyle=\color{blue},
    numberstyle=\tiny\color{codegray},
    stringstyle=\color{codepurple},
    basicstyle=\footnotesize\ttfamily,
    breakatwhitespace=false,
    breaklines=true,
    captionpos=b,
    keepspaces=true,
    numbers=left,
    numbersep=5pt,
    showspaces=false,
    showstringspaces=false,
    showtabs=false,
    tabsize=4
}

\begin{document}
%%%%%%%%%%%%%%%%%%%%%%%%%%%%%%%%%%%%%%%%%%%%%%%%%%%%%%%%%%%%%
%%%%%%%%%%%%%%%%%%%%%%%%%% COVER PAGE %%%%%%%%%%%%%%%%%%%%%%%
%%%%%%%%%%%%%%%%%%%%%%%%%%%%%%%%%%%%%%%%%%%%%%%%%%%%%%%%%%%%%
\begin{titlepage}

\begin{center}
\begin{minipage}{10cm}
	\begin{center}
	\textbf{UNIVERSITE DES SCIENCES ET DE LA TECHNOLOGIE USTHB}\\[0.1cm]
    \textbf{DEPARTEMENT D'INFORMATIQUE}
	\end{center}
\end{minipage}\hfill


\textsc{\Large }\\[2.5cm]
{\large \bfseries Project TP SPS}\\[1cm]

\textsc{\Large }\\[1cm]
\rule{\linewidth}{0.3mm} \\[0.4cm]
{ \huge \bfseries\color{darkblue} A Python Code Documentation for Genomic Sequence Analysis\\[0.4cm] }
\rule{\linewidth}{0.3mm} \\[3cm]


% Author and supervisor
\noindent
\begin{minipage}{0.4\textwidth}
  \begin{flushleft} \large
    \emph{Présenté par :}\\
    \textbf{GUELFEN Abdelheq}\\
  \end{flushleft}
\end{minipage}

\vfill
{\textbf{\large {Année universitaire} 2023-2024}}
\end{center}
\end{titlepage}
%%%%%%%%%%%%%%%%%%%%%%%%%%%%%%%%%%%%%%%%%%%%%%%%%%%%%%%%%%%%%
%%%%%%%%%%%%%%%%%%%%%%%%%% END PAGE %%%%%%%%%%%%%%%%%%%%%%%%%
%%%%%%%%%%%%%%%%%%%%%%%%%%%%%%%%%%%%%%%%%%%%%%%%%%%%%%%%%%%%%

\tableofcontents
\newpage

\section{Introduction}
This is a python code documentation about a program that automate the proccess of analysing sets of DNA.\\
Source code is seperated into 3 directories and 2 files:
\begin{itemize}
\item{\textbf{data}: data is the folder where our DNA sequences are stored}
\item{\textbf{modules}: This folder has three objects, each with a specific role}
\item{\textbf{output}: output is the folder where our results are stored}
\item{\textbf{main.py}: This is our main entry point to the program}
\item{\textbf{variables.py}: This file is where global variables are stored and will be useed throught the lifetime of the program}
\end{itemize}

\section{Main file (main.py)}
We are importing all the modules we need from dir:modules
\begin{lstlisting}[language=Python, style=PythonStyle]
from modules.dna_generator import DNA_Genrator
from modules.dna_processor import DNA_Processor
from modules.protein_builder import Protein_Builder

from variables import results
\end{lstlisting}
The function starts by showing an interactive menu with multiple choices to choose from.\\
User should choose a number from (1-11) according to the menu.
\begin{lstlisting}[language=Python, style=PythonStyle]
def main():
    print(
        """
 1: Generate a DNA
 2: Validate the DNA
 3: Calculate the necleotides base frequency
 4: Translate DNA to RNA
 5: Translate RNA into a protein
 6: Calculate the inverse complement of a DNA
 7: Calculate the taux of GC in a DNA
 8: Calculate the frequency of a codons in a DNA
 9: Make a random mutations to DNA
10: Find a motif in a DNA
11: Save results in a file
"""
    )
\end{lstlisting}
An infinite loop is used to keep the program running,
for every number chosen a specific function gets executed to solve that problem
\begin{lstlisting}[language=Python, style=PythonStyle]
while True:
    choice = input("> ")

    if choice == "quit()":
        break

    if choice == "1":
        DNA_GENRATOR().generate()

    if not results.get("DNA"):
        print("[ERROR]: Please generate a set of DNA first")
        continue

    if choice == "2":
        DNA().validate()
        print(f"DNA is validated: {results['is_dna_validated']}", "\n")

    if choice == "3":
        ...

\end{lstlisting}
The program can not manipulate DNA without generating a sequence first, so this condition limit and warn user to generate a sequence of DNA first
\begin{lstlisting}[language=Python, style=PythonStyle]
if not results.get("DNA"):
    print("[ERROR]: Please generate a set of DNA first")
    continue
\end{lstlisting}

\section{DNA generator(modules/dna\_generator.py)}
\subsubsection{Function generate()}
There are two type of generating a DNA sequence, either randomly or from a file.\\
The function prints a submenu and take a choice from the user and execute one of the two functions \textbf{generateRandomly()} or \textbf{chooseFromFile()}.\\
The input is limited to \textbf{(a, b or 0)} by using the \(assert\) keyword.\\
The user can choose \(0\) to go back to main menu

\begin{lstlisting}[language=Python, style=PythonStyle]
def generate(self):
    print(
        """a: generate a random set
b: choose from a predefined sets
0: go back to main menu
    """
    )

    while True:
        choice = input("(1: generate DNA)>> ")

        try:
            assert choice in ("a", "b", "0")

            if choice == "a":
                results["DNA"] = self.generate_randomly()
                print(results["DNA"], "\n")
                break

            if choice == "b":
                results["DNA"] = self.choose_from_file()
                print(results["DNA"], "\n")
                break

            if choice == "0":
                break

        except (AssertionError, ValueError):
            print("ERROR: Please enter a valid choice")
\end{lstlisting}

\subsubsection{Function generateRandomly()}
This function takes an input(number of sequences to generate) and return a DNA sequence.\\
We choose a Random necleotide and save it in a list for the number of choice entered by user, this has been implemented using list comprehension (simple way of writing for loop).\\
\textbf{''.join()} is used to trnasform a list to a string.
\begin{lstlisting}[language=Python, style=PythonStyle]
def generate_randomly(self):
    while True:
        print("how many sequence do you want?")
        choice = input("(1: generate DNA)>> ")

        try:
            assert choice.isdecimal

            return "".join([random.choice(SYMBOLS) for _ in range(int(choice))])
        except (AssertionError, ValueError):
            print("ERROR: Please enter a valid number\n")
\end{lstlisting}

\subsubsection{Function chooseFromFile()}
This function presents a menu of DNA sequences, reads them from a file, and allows the user to choose one by entering a corresponding number. It validates the input and returns the selected DNA sequence. If the input is invalid, it prints an error message and prompts the user to enter a valid number.

\begin{lstlisting}[language=Python, style=PythonStyle]
def choose_from_file(self):
    print(
        """1: sequence 1 consisting of 1000 bases.
2: sequence 2 consisting of 1000 bases.
3: sequence 3 consisting of 500 bases.
4: sequence 4 consisting of 100 bases.
5: sequence 5 consisting of 700 bases.
    """
    )

    sequences = self.read_fasta()

    while True:
        choice = input("(1: generate DNA)>> ")
        try:
            assert choice.isdecimal

            return sequences[int(choice) - 1]

        except (AssertionError, ValueError):
            print("ERROR: Please enter a valid number\n")
\end{lstlisting}

\end{document}
